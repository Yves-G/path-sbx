\documentclass[a4paper,10pt]{article}

\usepackage{hyperref}
\usepackage{mathpazo}
\usepackage{amsmath}
\usepackage{listings}
\usepackage{todonotes}

\usepackage{tikz}
\usetikzlibrary{calc}

\setlength{\oddsidemargin}{-0.4mm}    % 25 mm left margin - 1 in
\setlength{\evensidemargin}{\oddsidemargin}
\setlength{\topmargin}{-5.4mm}        % 20 mm top margin - 1 in
\setlength{\textwidth}{160mm}         % 20/25 mm right margin
\setlength{\textheight}{247mm}        % 20 mm bottom margin
\setlength{\headheight}{5.1mm}
\setlength{\headsep}{5mm}
\setlength{\parindent}{0mm}
\setlength{\parskip}{\medskipamount}

\title{0 A.D.\@ Formation Pathfinder Design}
\author{Yves Gwerder -- \url{https://github.com/Yves-G/path-sbx}}

\begin{document}

\maketitle

This document describes the current state of an experimental formation pathfinding implementation. The implementation currently runs in a sandbox environment in the web browser and uses javascript and other web technologies. In the future it's meant to be used for 0 A.D., an open source real time strategy game (see \url{http://play0ad.com}).

\tableofcontents

\section{Overview}

\subsection{Goals}

\begin{itemize}
 \item Movement of actors.
 \item Actors are single units or groups of units in different formations.
 \item Units can be infantry, cavalry or siege equipment.
 \item Ships are not considered yet. They will be kept as is.
 \item Movement appears sensible and intelligent.
 \item Movement behavior is predictable.
\end{itemize}

\subsection{Terms}

\begin{itemize}
 \item \textbf{Actors} --
  Single units or groups.
 \item \textbf{Groups} --
  Multiple units moving in a formation.
 \item \textbf{Formation} --
  A specific layout and parametrization influencing how a group moves (and fights).
\end{itemize}

\subsection{Layers}

\subsubsection{Path planning}
This part is about planning long range paths of actors, potentially across large distances.
Planned paths might have to be changed later because players can decide to do unpredictable things like moving other units, building walls or closing gates.
We might also not consider all aspects that could potentially influence the plans because of performance or complexity reasons.
This means that we might have to change a plan even if nothing unpredictable happens.
This cannot be completely avoided in practice.

\subsubsection{Steering}
Steering is about following the planned paths in a sensible way.
Path planning does not consider moving actors, but steering does.
Steering also manages arrangement in formation and movement of individual units as part of a formation.

\subsubsection{Locomotion}
Locomotion describes the low level aspects and specifics of actors and their movement.

\section{Path planning}

Path planning uses an A* algorithm over a 2D grid of navcells.
0 A.D. uses JPS optimizations, but the testing environment just uses plain A*.
Passability information in the grid gets generated from terrain passability information and from static obstructions.
Such static obstructions include buildings built by the player, objects placed by the map maker or ressources like trees or rocks.
The process to project these 3D objects to a 2D passability grid representation is called rasterization.
The testing environment just considers terrain passability and unit obstructions but does not do any rasterization of static obstructions.
Movable actors are currently not added to the passability grid because it's expected that they probably have moved away by the time the actor has reached their position.
On large maps it might take minutes until an actor has moved along a whole path.

It's planned to change the behavior of movable actors for those actors that are static (not currently moving) when the path is planned.
In the current implementation, units blocking a passage can prevent the pathfinder from finding a path around them.
This happens when the short range pathfinder doesn't find a way and has to invoke the long range pathfinder instead.
Because static units are currently not considered by the long range pathfinder, it finds the same blocked path again.

\section{Steering}

\subsection{Groups}
When units are moving in a group, the group takes over the path planning layer while units are responsible for steering and locomotion only.
On the steering layer, the group provides important information to its member units that allows them to walk in formation and avoid obstacles.

A flow field generated by the group allows units to avoid dead ends and obstacles and provides the necessary information for pathfinding-aware movement.
The group also calculates the spots where units should be according to the current formation, position and orientation.
 

The group defines:
\begin{itemize}
 \item The movement corridor for units in the group.
 \begin{itemize}
  \item The center spline of the corridor.
  \item The width of the corridor.
 \end{itemize}
 \item The available spots for member units.
 \item The flow field.
\end{itemize}

Tasks of the group are:
\begin{itemize}
 \item Provides a flow field for units in the group to use for steering towards the next waypoint.
This task comprises of the following sub-tasks:
 \begin{itemize}
  \item Trigger generation of the initial flow field.
  \item Trigger updates of the flow field.
  \item Define the size of the flow field.
  \item Generate the flow field.
 \end{itemize}
 \item Calculates the current center of all units in the formation.
 \item Coordinates movement with other groups.
\end{itemize}

\subsubsection{Group position}

Because there are multiple units in the group, the group itself doesn't have a clear position. We use two types of group positions.

\begin{itemize}
 \item \textbf{Average position} --
  Units far behind might have to move faster relative to other units to catch up.
  This position can be calculated by taking the average position of all units in the group and then taking the point on the path spline which is closest to that average position.
 \item \textbf{Move position} --
  The designated spots for units in the current formation need to be positioned relative to a group position, formation leader or similar.
  This position can be calculated by taking the average position and moving forward on the path spline some defined distance.
\end{itemize}

\todo[inline]{Because the path spline is currently just a list of vectors, the average position can "jump" quite a bit when there's a sharp turn from one vector to the next. 
A smooth curve would probably be a better representation than a list of vectors. }

When the average position reaches a vector, all previous vectors from the path spline are deleted.
This has the following advantages:
\begin{itemize}
 \item It reduces the "jumping" effect described in the todo.
  Otherwise, it could happen that the average position jumps backwards to the previous vector again.
 \item It makes it possible to just consider the first two vectors on the path instead of all vectors.
  This avoids issues where a vector further along the path is closer to the average position.
  This could otherwise happen when the path makes a slope around an obstacle.
\end{itemize}


\subsubsection{flow field}
Steering just based on passable and impassable cells is not sufficient because units can't detect dead ends. 
A similar problem occurs when units try to steer towards their designated spot in the formation.
That spot could be on impassable terrain or in a dead end.
Because the spot changes each turn, it also means that the shortest path from the unit to that spot can change completely.
This can cause the erratic movement behaviour like units changing their direction for no apparent reason. 
A flow field is used to solve these problems.
It calculates the length of the shortest path from each cell to a goal cell.
That goal cell is a waypoint on the central path spline.
The flow field gets calculated once for the group and stays the same over a large number of turns.
This ensures consistent movement.
For unit steering behaviour it's important that following the flow field has a high priority compared to other steering goals like moving to the designated spot in the formation.

\todo[inline]{Giving priority to different steering goals has to be situational.
Otherwise, flow field following would need such a high priority that it can cancel out all the other steering factors. 
This would make the other factors useless or at least very inefficient.}

Calculating the flow field for the whole map would be too slow.
It would be sufficient to calculate the flow field for cells inside the group's corridor because units aren't supposed to leave that corridor.
However, the current implementation uses a rectangular flow field region that covers a section of the group's long range path. 
This way the conversion from flow field navcell indexes to navcell indexes on the world sized grid is also straightforward.
Because only a section of the path is covered at a time, the flow field has to be rebuilt as the units move along the path.

\paragraph{Flow field region} ~\\

The vectors forming the path spline are important for calculating the flow field region.
They show us where the path goes and help us set the dimmensions of the region accordingly.
Just using these path vectors alone isn't sufficient, though.

We have to make sure that all unit positions are covered by the flow field.
The most obvious approach is to do exactly that. 

Take all units in the group and calculate a region that contains them all:

\begin{align*}
  leftX &= \text{min}(Unit0.x, Unit1.x \dots UnitN.x)\\
bottomZ &= \text{min}(Unit0.z, Unit1.z \dots UnitN.z)\\
 rightX &= \text{max}(Unit0.x, Unit1.x \dots UnitN.x)\\
   topZ &= \text{max}(Unit0.z, Unit1.z \dots UnitN.z)
\end{align*}

As an alternative solution, it was considered to keep a vector of the path until all units have passed it and then use the vectors only for flow field size calculation.
The difficulties here are figuring out when units have actually passed a vector.
This requires knowledge of passability and isn't trivial.

The next step is to expand the region size to contain additional vectors until a certain distance from the formation move position is reached.
Here's how the region size is expand with all the waypoints (ponints where the vectors point to).
The waypoints are represented as "WP".
It also uses the previously calculated bounds (leftX, bottomZ, rightX, topZ).

\begin{align*}
  leftX &= \text{min}(leftX, WP0.x, WP1.x \dots WPN.x)\\
bottomZ &= \text{min}(bottomZ, WP0.z, WP1.z \dots WPN.z)\\
 rightX &= \text{max}(rightX, WP0.x, WP1.x \dots WPN.x)\\
   topZ &= \text{max}(topZ, WP0.z, WP1.z \dots WPN.z)
\end{align*}

There must also be enough room perpendicular to the path spline for the formation to move.
We could try to only expand perpendicular to the spline and only where it's needed, but for simplicity we just expand on all sides:

\begin{lstlisting}[language=C++]
	corridorWidth = 30;
	leftX -= corridorWidth / 2;
	rightX +=  corridorWidth / 2;
	bottomZ -=  corridorWidth / 2;
	topZ += corridorWidth / 2;
\end{lstlisting}


\paragraph{Flow field generation trigger} ~\\

It's important to generate a new flow field early enough.
Obviously, units of the group must never leave the flow field.
Besides that, it's also important to update the flow field early enough before the waypoint is reached.
A good trigger is a minimum distance between the formation move position and the target position of the last vector used during the flow field size calculation.

\todo[inline]{Units shouldn't be too much ahead of the formation move position, so it should just work in most cases.
Still, it should be described exactly how to make sure that calculation of a new flow field gets triggered early enough.}

\subsection{Units}

\end{document}
